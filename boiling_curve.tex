% ============================================================================
% Japanese
% ============================================================================
\section{沸騰曲線}
沸騰曲線をFig. \ref{bc}(a)へ示す.
各定常点と時系列の擬沸騰曲線を示している.
また,各定常点を(1)-(13)と番号振った.
(1)から沸騰が始まり,$V$とともに$q$は増加する.
(5)から$V$を上げると,遷移域となり$q$が減少し$\Delta{}T_{\mathrm{sub}}$が増加する.
(7)以降では,MEBが発生し,$V$とともに$q$は増加する.
((1)-(5)間が核沸騰領域,(6)が遷移領域,(7)-(13)間がMEB領域である.)
沸騰曲線から本試験では典型的なS-MEBが発生していることがわかる(\cite{Horiuchi2021}).
MEB領域においては,(7)-(10)間と(10)-(13)間とでグラフの傾きが変化している.
この理由については後述する蒸気泡挙動の観察結果をもとに検討する.
Figure \ref{bc}(b)へFig. \ref{bc}(a)の縦軸を熱伝達率$h$としてを示す.
ここで,熱伝達率$h$を次式で定義する.
\begin{eqnarray}
  h
  &=& \dfrac{q}{T_{\infty}-T_{\mathrm{w}}}
  = \dfrac{q}{T_{\mathrm{sub}}+T_{\mathrm{sat}}}
\end{eqnarray}
遷移域(6)とMEB遷移直後(7)-(8)では熱伝達率は低下する.
(9)-(13)間ではCHFに対応する点を超える熱伝達率を示している.


\begin{figure}[btp]
  \centering
  \includegraphics[width=\textwidth]{./fig2/boiling_curve/boiling_curve_tot_1.pdf}
  \caption{Pseudo-boiling curve with fully-developed points under $\Delta T_{\mathrm{sub}}~=~40~\mathrm{K}$}
  \label{bc}
\end{figure}
% ============================================================================



% ============================================================================
% English
% ============================================================================




% ============================================================================
